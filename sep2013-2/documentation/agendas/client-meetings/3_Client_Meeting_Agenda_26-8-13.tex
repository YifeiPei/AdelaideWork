%%%%%%Agenda for client meeting Monday 26 August 2013. Prepared by Bowen Tao 24/08/13.
\documentclass[11pt, a4paper]{article}
\usepackage{times}
\usepackage{ifthen}
\usepackage{amsmath}
\usepackage{amssymb}
\usepackage{graphicx}
\usepackage{setspace}

%%% page parameters
\oddsidemargin 0 cm
\evensidemargin 0 cm
\textwidth 16 cm
\topmargin -2.5 cm
\textheight 25.5 cm

\renewcommand{\baselinestretch}{1.4}\normalsize
\setlength{\parskip}{0pt}


\begin{document}

%%%mention the number, time, and venue of the meeting
\noindent The {\em third} client meeting for Software Engineering Group 2. {\em Weekly} project meeting will be held in {\bf BCS Software Lab, Room 462, Ingkarni Wardli} at {\bf 2:30pm every Monday}.


\vspace*{15pt}

\begin{center}
\huge \bf Agenda
\end{center}



\section*{Chair: Bowen Tao \large (a1622211)}

\vspace*{15pt}

%%%if some students cannot make the meeting due to some reasons, their names should appear here.
%%%\section{Apologies}
%%%Student name and ID number

%%%short presentation about the work of previous week or any milestone specified in the course.
\section{Presentation}

\begin {enumerate}
\item Jianqiu Li (1635717) will present to the client the Software Requirements Specification (SRS).

\item Yu Hong (1616861) will show the new prototype of the robot and ask for suggestions.

\item Yifei Pei(1611648) will show the GUI design and ask for suggestions.
\end {enumerate}
%%%any schedules for this meeting should go next, each with a separate section.
%%%for example, the first meeting is about requirement elicitation, like the following.

%%%if there are more subissues, make them as subsections.
\section{Milestone Negotiation}

The whole group will negotiate the milestones for Week 9 and Week 10 with the client for the rest of the meeting. 

\section{Further Elicitation of Requirements}

\begin{enumerate}

\item Question for the DTD, where should the (0,0) point position? Top left or bottom left? (Yifei Pei 1611648)

\item DTD stated that obstacles can be presented by point. How do the obstacles act
as a point physically? How do we differentiate road intersection from point shaped
obstacles? We have a plan to use different colours for them, will it work? (Bowen Tao 1622211) 

\item From the DTD example, the two roads are perpendicular to each other, but we still
need to ask exceptions. Do they have constant turning angles or not? (Matthew Nestor 1132338)  

\end{enumerate}

%%%finally, specifies time of next meeting
\section{Date of Next Meeting}
\noindent Next meeting to be held on Monday 2 September 2013 at 2:30pm.


\end{document}
