%%%%%%Agenda for client meeting Monday 14 October 2013. Prepared by Abdulaziz Alhulayfi 12/10/13.
\documentclass[11pt, a4paper]{article}
\usepackage{times}
\usepackage{ifthen}
\usepackage{amsmath}
\usepackage{amssymb}
\usepackage{graphicx}
\usepackage{setspace}

%%% page parameters
\oddsidemargin 0 cm
\evensidemargin 0 cm
\textwidth 16 cm
\topmargin -2.5 cm
\textheight 25.5 cm

\renewcommand{\baselinestretch}{1.4}\normalsize
\setlength{\parskip}{0pt}


\begin{document}

%%%mention the number, time, and venue of the meeting
\noindent The {\em eighth} client meeting for Software Engineering Group 2. {\em Weekly} project meeting will be held in {\bf BCS Software Lab, Room 462, Ingkarni Wardli} \textnormal{at} 2:30pm  \textnormal{on} Monday 14 October 2013.


\vspace*{15pt}

\begin{center}
\huge \bf Agenda
\end{center}



\section*{Chair: Abdulaziz Alhulayfi \large (a1642362)}

\vspace*{15pt}

\section{Absence}

Yifei Pei (1611648) has apologised for not attending the client meeting due to his illness.

%%%if some students cannot make the meeting due to some reasons, their names should appear here.
%%%\section{Apologies}
%%%Student name and ID number

%%%short presentation about the work of previous week or any milestone specified in the course.	
\section{Presentation}

\begin {itemize}
\item  Yu Hong (1616861) is going to present the testing techniques which the group is going to apply to this project.
\end {itemize}

\section{Milestone for Week 10}

\subsection{GUI Features}
\begin {enumerate}

\item Synchronisation of the robot movement on GUI

\begin{itemize}
\item The robot can be shown on GUI when it is doing Y axis movement
\end{itemize}

\item Map and GUI redesign

\begin{itemize}
\item The starting point is now located at the bottom left
\item New representation of closure, obstacle and unexplored zone
\end{itemize}

\end {enumerate}

\subsection{AI Mode of The Robot}
\begin {enumerate}
\item Basic AI movement such as following the road and intersection identification.
\end {enumerate}


%%%any schedules for this meeting should go next, each with a separate section.
%%%for example, the first meeting is about requirement elicitation, like the following.

%%%if there are more subissues, make them as subsections.


%%%finally, specifies time of next meeting
\section{Date of Next Meeting}
\noindent Next meeting to be held on Monday 21 October 2013 at 2:30pm.


\end{document}
