% Group: 	2
% Name: 	Coding Pharaohs
% Document: 	Second and final draft of the minutes of the first client meeting
% Author:	Matthew Nestor
% Date:		Saturday August 18 2013

\documentclass[12pt, a4paper]{article}
\topmargin -1cm
\title{Minutes of the First Client Meeting}
\author{\textsc{Coding Pharaohs} (Group 2)}
\date{August 12, 2013}
\begin{document}
\maketitle
\begin{tabbing}
\textbf{Chair}~~~~~~~~\=Yifei Pei           \\
\textbf{Secretary}    \>Matthew Nestor      \\
\textbf{Members}      \>Abdulaziz Alhulayfi \\
                      \>Bowen Tao           \\
                      \>Jianqui Li          \\
                      \>Yu Hong             \\
\textbf{Apologies}    \>None                \\
\end{tabbing}
\section{Time and Place}
The first client meeting for Software Engineering and Project was held in \textbf{Ingkarni
Wardli, Room 462 \textnormal{at} 2:20pm \textnormal{on} Monday 12 August 2013}.
\section{Quorum Announcement}
Having determined that quorum was satisfied, the Chairman Mr.~Pei declared the meeting open.
\section{Summary of Previous Meeting}
Not applicable.
\section{Group Milestone: Poster}
\subsection{Overview}
Mr.~Pei presented the group poster the team created to advertise itself.
He explained the reasoning behind various design decisions, as well as the reasoning behind
individual members' group roles.
\subsection{Detailed Presentation}
\begin{itemize}
 \item Mr.~Pei explained the rationale behind the group's name, \emph{Cod\-ing Phar\-aohs}.
First, it is a strong name, implying that the group consists of highly proficient coders.
Second, there is a link between Ancient Egypt and the Robot Project the group is undertaking.
The project is to design and programme a robot to safely navigate a d\'ebris-stricken city and
mark out hazardous zones therein. Robots have a similar role in ancient Egyptian pyramids: 
they explore and mark out potentially hazardous areas and boobytraps.
 \item As the team finished the poster well before the deadline, a second poster was made. 
There was some debate as to which of the posters was preferable. The team asked the client 
which poster he prefered, but as there is a competition for the best poster, he politely 
declined to influence our decision. In the end, the group chose the first poster.
 \item The poster contained information about each team member, including the group roles
for which they were elected. Thus, Mr.~Pei used the poster presentation to introduce the group
members' roles.
They were as follows:
\begin{tabbing}
~~~~~~~~~~~~\=Abdulaziz Alhulayfi~~~~~~~~\=Documentation Manager  \\
            \>Bowen Tao                  \>Quality Control Manager\\
            \>Jianqui Li                 \>Hardware Manager       \\
            \>Matthew Nestor             \>Software Manager       \\
            \>Yifei Pei                  \>Project Manager        \\
            \>Yu Hong                    \>Testing Manager        \\
\end{tabbing}
\end{itemize}
\section{Individual Milestone Reports}
No individual milestones were reported.
\section{Project Administration}
The lecturer made some points about project administration:
\begin{itemize}
 \item The group SVN repository is now available.
 \item The robot and lab are now available.
 \item The DTD document and examples will be available early next week.
 \item Like this meeting, the next meeting, to be held next week, will focus on requirements collection.
       However, the client will be a different person from this week.
\end{itemize}
Mr.~Pei noted a few points of administration:
\begin{itemize}
 \item Mr.~Pei shall shortly create and set out the folder structure of the group SVN repository.
 \item All members shall familiarise themselves with \LaTeX and the SVN repository.
 \item Mr.~Nestor shall write up the minutes for this meeting in \LaTeX.
 \item Mr.~Alhulayfi shall write up the agenda for the following meeting in \LaTeX.
\end{itemize}
The lecturer agreed to be contactable via email for follow-up questions after client meetings. Mr.~Li, Mr.~Nestor,
and Mr.~Hong agreed to convene after the meeting to construct a prototype of the robot body.
\section{Requirements Elicitation}
The majority of the meeting consisted of interviewing the client for project requirements. This
was driven by a list of questions compiled by Mr.~Pei, as well as spontaneous questions from
other members. A paraphrased summary of the questions, and client answers, follows:
\subsection{Functional Requirements}
\begin{enumerate}
 \item \textbf{Mr.~Pei:}       Is the robot intended for use in one specific site, or for use in multiple sites? \\
       \textbf{Client:}        It is for use in just one site, the XML file for which will be issued soon.
 \item \textbf{Mr.~Pei:}       Is the software intended for use in one or many robots? \\
       \textbf{Client:}        Just one robot.
 \item \textbf{Mr.~Pei:}       To what extent should the robot's operation be automated? Does the robot need to move
                               with or without an operator? Does the robot need the operator's command to mark road closures? \\
       \textbf{Client:}        The system should support both manual and automatic modes. Take the following
                               scenario. In the beginning, a truck will deliver the robot to the edge of the city, for
                               consider that the robot in reality may be quite large. The robot should begin in automatic mode
                               and follow the road to the dangerous areas in order to mark the road closures. Manual mode is
                               crucial from a safety standpoint. For example, if the robot detects an obstacle along its path,
                               it should be stopped, and it should wait for operator to issue commands.
 \item \textbf{Mr.~Nestor:}    So should that stopping occur automatically upon detection of an obstacle? \\
       \textbf{Client:}        That's correct. Basically, the robot will have a bump sensor to detect obstacles.
                               In natural disasters this will always happen; buildings and other objects will block the
                               road and you will have to choose another path to get where you want to go.
 \item \textbf{Mr.~Nestor:}    And at that point, the new path is determined by the user, rather than the robot? Or should
                               the robot be able to navigate a new path automatically? \\
       \textbf{Client:}        That is one possible solution. Here's an example. In the description we mention that, for
                               safety reasons, the robot should always stay on the road; it should never run off the road.
                               Otherwise it would be very dangerous: the robot could hit buildings or people. In reality,
                               there is always a possibility that the robot will go off the road. In this case, firstly, the
                               robot should automatically stop, and wait for manual control from the operator. The operator
                               can then move the robot back to the road. There are many other situations in which manual
                               control is necessary. For example, what happens when communication is lost between the robot
                               and the host system?
 \item \textbf{Mr.~Pei:}       What should happen once the robot has searched the entire area? \\
       \textbf{Client:}        The robot should return to the starting point.
 \item \textbf{Mr.~Pei:}       So the robot should automatically find a path, search the whole city, and when it has finished,
                               it should return to the starting point? \\
       \textbf{Client:}        You will already know that something has happened in a particular area of the city. You will
                               know a point in the city where the disaster is thought to have occured, and there will be a
                               radius around that point. At any point that a road intersects that radius, a road closure should
                               be marked. This is the main task of the robot, so it may not be necessary for it to explore the
                               whole city.
 \item \textbf{Mr.~Nestor:}    So you're saying that the robot isn't required to make a complete map of the city, only the areas
                               of the city that have been damaged? \\
       \textbf{Client:}        Yes.
 \item \textbf{Mr.~Li:}        So, when the robot meets an obstacle, it needs to record the place of the obstacle? \\
       \textbf{Client:}        Yes. When the robot detects an object, it should mark it, not the physical map, but on the host
                               screen.
 \item \textbf{Mr.~Nestor:}    In the damaged area, should we assume that there are some circumstances under which the robot
                               must go off-road? \\
       \textbf{Client:}        The robot should never leave the road. The robot is not allowed to go into the damaged areas; it
                               simply finds the edge of the affected areas.
 \item \textbf{Mr.~Pei:}       How does the robot identify the damaged area? \\
       \textbf{Client:}        The robot has prior knowledge of the affected areas.
 \item \textbf{Mr.~Nestor:}    How detailed is our knowledge of the city and the affected areas? \\
       \textbf{Client:}        You will be given a point and a distance around it, for example one hundred metres around that
                               that point, which represents the dangerous zone. The main task is then to determine which streets
                               intersect that zone, and mark them.
 \item \textbf{Mr.~Li:}        Will the obstacles be the same size? \\
       \textbf{Mr.~Pei:}       Yes, what sorts of shapes and sizes will the obstacles be? \\
       \textbf{Client:}        You can represent the obstacles as you like. Here are some other requirements I can tell you.
                               The city will be a scaled down model made out of A1 or A0 paper. The width of the streets will
                               be 5mm. Intersections will be a solid circle. So, when the robot arives at an
                               intersection, it detects a circle, and thereby knows it is at an intersection. Then, you need
                               to think about how to represent obstacles. If you represent obstacles with circles, then you'll
                               run into trouble.
 \item \textbf{Mr.~Nestor:}    Perhaps I misunderstood, but I thought the question was, How will the objects be represented
                               in the physical domain that the robot is trying to navigate? On the paper. For I assume
                               the robot doesn't know beforehand where the obstacles will be? \\
       \textbf{Client:}        It's half-half: you will know where some of the obstacles are, and not others, just like in reality.
                               It's a dynamic situation: yesterday there was no obstacle here, but today there might be. \\
       \textbf{Mr.~Nestor:}    So what sort of form will these obstacles take? \\
       \textbf{Client:}        You mean, on the paper? \\
       \textbf{Mr.~Nestor:}    Yes. \\
       \textbf{Client:}        Any solid object. Imagine this object is here [client points to mobile phone]. The robot has a
                               bump sensor which will detect it. After it detects it, it has to mark it on the host side, using
                               a different shape.
 \item \textbf{Mr.~Alhulayfi:} How will the sensors recognise shape, whether something is an the obstacle or an intersection? \\
       \textbf{Client:}        You will have the light sensor for this. So you will need to figure out how to use the light
                               sensor to detect whether something is a solid line, a circle, or a rectangle.
 \item \textbf{Mr.~Pei:}       Will the user be on site with the robot, or will it be remotely controlled? \\
       \textbf{Client:}        It will be operated remotely; the user will be in an office far away, potentially in another city.
                               This has implications for the whole project. So, the communication between the robot and the host
                               should happen in real-time. 
 \item \textbf{Mr.~Pei:}       Given communication must occur in real time, what sort of delay is acceptable? \\
       \textbf{Client:}        Good question. This is a safety-critical project. Acceptable delay will be less than 0.5 seconds.
 \item \textbf{Mr.~Pei:}       What kind of graphic user interface will the operator prefer? \\
       \textbf{Client:}        This question is too vague. \\
       \textbf{Mr.~Pei:}       Okay. Do you prefer a GUI or command-line control? \\
       \textbf{Client:}        Definitely a GUI, because a command-line has too much text to know what's happening quickly.
                               The GUI will have a map so the operator can see what's happening very quickly.
 \item \textbf{Mr.~Pei:}       How simple do you want the GUI? Will it be operated by an expert, or will many people need to
                               operate it? \\
       \textbf{Client:}        This is a technical question that will be covered in a later meeting. You will design a GUI and
                               show it to the client for feedback. This will happen in week 5 or 6, I think.
 \item \textbf{Mr.~Pei:}       We've discussed the representations of the dangerous areas, the roads, the obstacles. Is there
                               anything else, except for the roads and obstacles, such as buildings or walls? \\
       \textbf{Client:}        You don't have to worry about buildings because we assume the robot always stays on the roads.
 \item \textbf{Mr.~Nestor:}    Are we going to be given the map of the road as a data-structure first? Or do we have to determine
                               the map of the road on the basis of the lines on the paper? \\
       \textbf{Client:}        We give you a DTD, which specifies how to represent the city, the obstacles, the dangerous areas.
                               We also give you an example. With that example you can draw an initial map.
\end{enumerate}
\subsection{Safety Requirements}
\begin{enumerate}
 \item \textbf{Mr.~Pei:}       What action should the robot take when power is low or lost? \\
       \textbf{Client:}        As safety is critical, in such situations the robot should automatically stop and await further
                               instruction.
       \textbf{Mr.~Pei:}       Should it give a warning to the operator? \\
       \textbf{Client:}        Yes. If the robot's power falls below a certain level, you should give a warning.
 \item \textbf{Mr.~Pei:}       What action should the robot take if communication between the operator and the robot is lost? \\
       \textbf{Client:}        It should stop.
 \item \textbf{Mr.~Tao:}       Should the robot be expected to change automatically between automatic and manual modes? \\
       \textbf{Client:}        You mean, when the robot loses communication with the host? \\
       \textbf{Mr.~Tao:}       Yes. \\
       \textbf{Client:}        No, the robot should just stop. Otherwise you'll have trouble; the robot may kill people. Keep in
                               mind that the robot has survelance equipment, such as a camera. So whenever the robot travels
                               somewhere, the operator should see a video feed of the surrounding area. But if this feed is lost,
                               you don't know what the robot is doing, so it's very dangerous.
 \item \textbf{Mr.~Nestor:}    So we'll have three different types of sensors: the bumper sensor, light sensor, and a camera? \\
       \textbf{Client:}        You don't have a camera; it is just assumed.
 \item \textbf{Mr.~Pei:}       What is a safe force with which the robot may make contact with external objects, such as
                               obstacles? \\
       \textbf{Client:}        This is an important safety issue, for if the robot make contact with objects with too much force,
                               it could be damaged. For this reason, the robot should not move very fast; use a slow speed. \\
       \textbf{Mr.~Pei:}       How do you define the speed limit? What is a slow speed. \\
       \textbf{Client:}        You should be able to figure this out for yourselves by experimentation.
 \item \textbf{Mr.~Pei:}       How do you define a safe starting position? \\
       \textbf{Client:}        As the requirements document specifies, you can assume that a truck will deliver the robot to
                               the starting position. \\
       \textbf{Mr.~Pei:}       What if the robot starts in a dangerous zone? \\
       \textbf{Client:}        In the case, the truck couldn't have delivered it. So you can assume that the robot will start
                               in a safe area.
 \item \textbf{Mr.~Pei:}       Is the starting area randomly picked? \\
       \textbf{Client:}        Yes, but definitely inside the city, not outside.
       
\end{enumerate}
\section{Meeting Feedback}
The lecturer gave the group some feedback for its performance in its first client meeting. The following is a
summary of the key points.
\begin{itemize}
 \item The chairperson role is a managing role. This includes the allocation of tasks, making sure the meeting finishes on time,
       and ensuring all items are covered. In this way, the chair doesn't have to do everything.
 \item Recording minutes is good, but you should also write notes for the minutes as the meeting unfolds. Otherwise, you have
       to relisten to everything recorded in order to start writing. If you already have hand-written notes, you merely need
       the recording as a reference.
 \item The group still hasn't asked all the requirements questions necessary, and should carefully about what to ask for the next
       meeting.
\end{itemize}

\section{Adjournment}
The next meeting is a group meeting to be held in the same place, namely Ingkarni
Wardli, Room 462 at 2pm on Thursday 15 August 2013.\\
The meeting closed at 3:12pm.
\end{document}
