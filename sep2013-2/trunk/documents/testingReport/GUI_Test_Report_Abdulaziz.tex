\documentclass[titlepage]{article}
%\usepackage[T1]{fontenc}
%\usepackage[utf8]{inputenc}
%\usepackage{lmodern}
%\usepackage{babel}
\usepackage{amssymb}
\usepackage{amsmath}
\usepackage[table]{xcolor}
\usepackage{booktabs}
\usepackage[hidelinks]{hyperref}
\usepackage{tabularx}
\usepackage{graphicx}
\usepackage{acronym}
\usepackage{float}
\usepackage[font=normal, labelfont = bf]{caption}
\usepackage{sidecap}
\usepackage{rotating}
\usepackage{rotfloat}
\usepackage[ampersand]{easylist}
\usepackage{multirow}

%%% page parameters
\oddsidemargin 1 cm
\textwidth 14cm
\topmargin -1 cm
\textheight 23 cm

\setcounter{tocdepth}{2}
\title{GUI Testing Report}
\author{Group 2   {\em Coding Pharaohs}\\ 
\\
Presented by Abdulaziz ALHULAYFI a1642362} 
\date{\today}
\begin{document}
\maketitle

\pagebreak

%%INTRODUCTION
\section{Introduction}
This document provides manual testing of the Graphical User Interface (GUI) that is used to control the whole program of the robot. The document also shows all testing outputs of the functionalities that can be performed using the GUI such as connect, move forward, mark road closure etc.. 

\section{Test Cases}
\subsection{GUI Connection Buttons}
\begin{tabular}{|l | p{13cm} | }
    \hline
    Description & Connect and disconnect buttons can establish a connection as well as disconnection        between the PC and the robot  \\ \hline
    Priority & High  \\ \hline
    Prerequisites & The robot is turned on and the program is run  \\ \hline
    Test procedure &  \begin{enumerate}
\item Run the program to open the graphical user interface.
\item Click on the "Connect" button to establish a connection.
\item Wait for few seconds for the PC to be connected to the Robot
\item Click on the "Disconnect" button to disconnect the PC from the robot
\end{enumerate}  \\ \hline
    Expected result & PC connects/disconnects with the robot and the message "Connected" appears on the the robot LCD display and GUI connection bar.  \\ \hline
    Actual result & PC connects/disconnects successfully with the robot  \\ \hline
    Status & Pass  \\ 
    \hline
  \end{tabular}

\subsection{Manual Movement Control Buttons}
\begin{tabular}{|l | p{13cm} | }
    \hline
    Description & In manual mode, two movement buttons (move forward, move backward) allow robot to move according to arrows directions. Two rotating buttons change the direction of the robot accordingly.\\ \hline
    Priority & High  \\ \hline
    Prerequisites & The robot is turned on and PC is connected to the robot  \\ \hline
    Test procedure &  \begin{enumerate}
\item Connect PC to the robot
\item Click on forward/backward pointed arrows buttons to move the robot forward or backward.
\item Click on right/left rotating buttons to change the direction of the robot according to arrows directions.
\end{enumerate}  \\ \hline
    Expected result & The robot moves forward and backward and it rotates according to arrows directions  \\ \hline
    Actual result & Robot moves and rotates appropriately  \\ \hline
    Status & Pass  \\ 
    \hline
  \end{tabular}

\pagebreak

\subsection{File Options Menu}
\begin{tabular}{|l | p{13cm} | }
    \hline
    Description & The file menu has three options: load, save and quit. These options are used to load xml map files, save cureently explored map to xml file and quit the whole program.\\ \hline
    Priority & High  \\ \hline
    Prerequisites & The program is run  \\ \hline
    Test procedure &  \begin{enumerate}
\item Run the graphical user interface
\item Click on the file options menu
\item Click on load and chose the xml file map from the computer
\item Click again on the file options menu and type a name for the current map
\item Click save to save the map to xml file
\item Click again on the file options menu and click quit to close the program
\end{enumerate}  \\ \hline
    Expected result & The map can be loaded/saved and the program closes when choosing quit\\ \hline
    Actual result & The map is successfully loaded and saved and the program is closed  \\ \hline
    Status & Pass  \\ 
    \hline
  \end{tabular}

\subsection{Auto Mapping Option Button}
\begin{tabular}{|l | p{13cm} | }
    \hline
    Description & When choosing the automatic mapping option, the robot starts to move and explore areas on the map independently\\ \hline
    Priority & High  \\ \hline
    Prerequisites & The program is run and PC is connected to the robot\\ \hline
    Test procedure &  \begin{enumerate}
\item Run the user interface
\item Connect PC to the robot
\item Place the robot on the physical map
\item Click on the "Start Auto Mapping" button
\end{enumerate}  \\ \hline
    Expected result & The robot runs in the automatic mode and it moves and explores the map independently\\ \hline
    Actual result & The robot successfully explores the map automatically \\ \hline
    Status & Pass  \\ 
    \hline
  \end{tabular}

\subsection{Mark Road Closure Button}
\begin{tabular}{|l | p{13cm} | }
    \hline
    Description & Marking road closures is a function that allows the robot to lower down the marker pen to mark road closure on the physical map. It also shows the road closure on the GUI map.\\ \hline
    Priority & High  \\ \hline
    Prerequisites & The program is run and PC is connected to the robot\\ \hline
    Test procedure &  \begin{enumerate}
\item Run the user interface
\item Connect PC to the robot
\item Place the robot on the physical map
\item Chose any type of exploration (manual or automatic)
\item Click on the "Mark Road Closure" button to mark road closure on both physical and GUI map
\end{enumerate}  \\ \hline
    Expected result & The robot runs lowers down its marker on the paper map and mark a closure and a closure appears on the GUI map\\ \hline
    Actual result & Road closure has been marked successfully \\ \hline
    Status & Pass  \\ 
    \hline
  \end{tabular}

\subsection{Stop Button}
\begin{tabular}{|l | p{13cm} | }
    \hline
    Description & Stop is a function in the user interface that stops all activities of the robot \\ \hline
    Priority & High  \\ \hline
    Prerequisites & The program is run and PC is connected to the robot, and the robot is moving \\ \hline
    Test procedure &  \begin{enumerate}
\item Run the user interface
\item Connect PC to the robot
\item Chose any type of exploration (manual or automatic)
\item When robot starts moving, click on the "Stop" button
\end{enumerate}  \\ \hline
    Expected result & When "Stop" button is clicked the robot should be terminated\\ \hline
    Actual result & The robot responded to the event and stoped all movements successfully \\ \hline
    Status & Pass  \\ 
    \hline
  \end{tabular}

\subsection{Return to Base Button}
\begin{tabular}{|l | p{13cm} | }
    \hline
    Description & Return to base is a function that allows the robot to return to its starting point \\ \hline
    Priority & Medium/High  \\ \hline
    Prerequisites & The program is run and the robot is in the process of exploring \\ \hline
    Test procedure &  \begin{enumerate}
\item Run the user interface
\item Connect PC to the robot
\item Chose any type of exploration (manual or automatic) for the robot to move from its starting point
\item Click on the "Return TO Base" button
\end{enumerate}  \\ \hline
    Expected result & The robot stops exploration and moves towards its starting\\ \hline
    Actual result & The robot returns to the base where it started \\ \hline
    Status & Pass  \\ 
    \hline
  \end{tabular}

\subsection{AI View Button}
\begin{tabular}{|l | p{13cm} | }
    \hline
    Description & AI view function displays the paths that the robot follows, and the spots of road closures, obstacles and intersections \\ \hline
    Priority & Medium  \\ \hline
    Prerequisites & The program is run and map is loaded \\ \hline
    Test procedure &  \begin{enumerate}
\item Run the user interface
\item Load xml map file
\item Click on the "AI View" button
\end{enumerate}  \\ \hline
    Expected result & AI view appears on the map with all details displayed \\ \hline
    Actual result & AI view is on and with the appropriate details displayed \\ \hline
    Status & Pass  \\ 
    \hline
  \end{tabular}

\subsection{Robot Speed Slider}
\begin{tabular}{|l | p{13cm} | }
    \hline
    Description & This function allows the robot to increase and decrease its speed \\ \hline
    Priority & Low/Medium  \\ \hline
    Prerequisites & The program is run and PC is connected to the robot, and the robot is moving \\ \hline
    Test procedure &  \begin{enumerate}
\item Run the user interface
\item Connect PC to the robot
\item Chose any type of exploration (manual or automatic) for the robot to move
\item Slide the speed slider right/left to increase or decrease the robot speed
\end{enumerate}  \\ \hline
    Expected result & The robot movement speed changes according to selected speed \\ \hline
    Actual result & The robot speed changes successfully \\ \hline
    Status & Pass  \\ 
    \hline
  \end{tabular}

\subsection{Robot Information Display}
\begin{tabular}{|l | p{13cm} | }
    \hline
    Description & Information display panel shows three types of information: Distance between robot and objects in the front of it, battery level and connection status \\ \hline
    Priority & High  \\ \hline
    Prerequisites & The program is run and PC is connected to the robot\\ \hline
    Test procedure &  \begin{enumerate}
\item Run the user interface
\item Connect PC to the robot
\item Place the robot on the physical map
\end{enumerate}  \\ \hline
    Expected result & All information is displayed \\ \hline
    Actual result & Robot information is displayed successfully \\ \hline
    Status & Pass  \\ 
    \hline
  \end{tabular}


\end{document}
