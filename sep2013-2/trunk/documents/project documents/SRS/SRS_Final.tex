\documentclass[titlepage]{article}
\usepackage{tabularx}
\usepackage{listings}
\usepackage{amssymb}
\usepackage{amsmath}
\usepackage[table]{xcolor}
\usepackage{booktabs}
\usepackage[hidelinks]{hyperref}
\usepackage{graphicx}
\usepackage{acronym}
\usepackage{float}
\usepackage[font=normal, labelfont = bf]{caption}
\usepackage{sidecap}
\usepackage{rotating}
\usepackage{rotfloat}
\usepackage[ampersand]{easylist}
\usepackage{multirow}

\lstset{breaklines=true} 

%%% page parameters
\oddsidemargin 1 cm
\textwidth 14cm
\topmargin -1 cm
\textheight 23 cm

\setcounter{tocdepth}{2}
\title{Software Requirements Specification for Software Engineering and Project}
\author{Group 2   {\em Coding Pharaohs}\\ 
\\
Abdulaziz ALHULAYFI a1642362  \\
Yu HONG	 a1616861  \\
Jianqiu LI a1635717  \\
Matthew NESTOR a1132338  \\
Yifei PEI a1611648  \\
Bowen TAO a1622211}

\date{\today \\ Final Document}
\begin{document}
\maketitle
\tableofcontents

\pagebreak
\section*{Revision History}
\begin{tabularx}\linewidth{|l|c|X|r|}
\hline
Name & Date & Reason For Changes & Version \\
\hline
\hline
Yu hong,JianQiu Li & 22/08/2013 & Initial Version & 0.1 \\
\hline
Yu hong & 24/08/2013 & Minor Modification & 0.2 \\
\hline
Yifei Pei & 25/08/2013 & Minor Modification & 0.3 \\
\hline
Abdulaziz Alhulayfi & 25/10/2013 & Modification & 0.4 \\
\hline
Abdulaziz Alhulayfi & 03/11/2013 & Adding New System Features and Use Cases & 1.0 \\
\hline
\end{tabularx}

\pagebreak
% section 1 Introduction

\section{Introduction}
\subsection{Purpose}
This document covers the specification of software requirements for developing the
SEP project in semester 2, 2013. The project objective is
to develop a Robot that is capable of marking road closures and identifying 
dangerous areas such that people are alerted not to move into these areas.
The project specifications and software requirements specified in this document are collected from the clients.\\
\\
The scope of this document is to specify the requirements that serves both the host program 
operation and the robot. The robot should be able to detect dangerous areas and send
 the information back to the host, and the operator at the host side is able to control the movement of the robot.

\subsection{Document Conventions}
The specification of the software requirements consists of functional requirements, nonfunctional requirements, and security requirements. Each requirement will be classified into High, Medium, or Low priority, which will be tagged in its corresponding requirement statement. Priority classification indicates the importance of the requirement to the project success
\subsection{Intended Audience and Reading Suggestions}
This document is intended for all stakeholders of the SEP project in 2013, which include clients, who are the users of the system, and all members of group two in this class, who are contributing to the project.\\
\\
The below sections contain the specification of software requirements which consist of functional requirements, nonfunctional requirements, and user requirements for both the robot and host systems, which are important information for the designers, developers and testers. Designers of the system need this information to design the architecture of the system, which is a key procedure before developers could start their work. Developers of the system also need these information to guide them implementing specific functions. Furthermore, testers need to test the specific functions of the system with accordance to these provided requirements.\\
\\
Since this document contains all the requirements that describe the functionalities of the system, it is recommended that this document should be thoroughly read.
\subsection{Project Scope}
The project scopes are to produce a prototype of the robot required in SEP project, and to deliver the corresponding documents according to the schedule.\\
\\
The robot should be able to identify and mark the road closures as well as dangerous areas in the map on both the robot and host side. To achieve this goal, a whiteboard marker would be mounted to the robot. Thus, whenever the robot marks on the map, the GUI (i.e., Graphical User Interface) on the host system would dynamically display the changes, with an XML (i.e., Extensible Markup Language) document being generated and updated. In addition, the robot should be able to automatically conduct the map by means of all possible features detected.\\
\\
With regard to the documentation of this project, SRS, SDD, SPMP, and user manual will be created according to the schedule of the project and maintained through the process of developing phase.\\
\subsection{References}


\begin{thebibliography}{9}

\bibitem{Project description} Project description\\
\label{Project description}
\begin{lstlisting}
Sheng, M 2013 , "Project Description", <https://cs.adelaide.edu.au/users/third/sep/13s2-sep-Adelaide/project/ProjectDescription.pdf>
\end{lstlisting}

\bibitem{1_ Client's meeting} Client's meeting minutes\_12-08-13\\
\label{1_ Client's meeting}
\begin{lstlisting}
Pei, Y 12/08/2013, https://version-control.adelaide.edu.au/svn/sep2013-2/documentation/minutes/client-meetings/1_client_meeting_minutes_12-08-13.pdf
\end{lstlisting}

\bibitem{2_ Client's meeting} Client's meeting minutes\_20-08-2013\\
\label{2_ Client's meeting}
\begin{lstlisting}
https://version-control.adelaide.edu.au/svn/sep2013-2/documentation/minutes/client-meetings/2_client_meeting_minutes_20-08-2013.pdf
\end{lstlisting}




\bibitem{Lego tutorial} Lego tutorial\\
\label{Lego tutorial}
\begin{lstlisting}
http://lejos.sourceforge.net/nxt/nxj/tutorial/
\end{lstlisting}


\bibitem{Blue-tooth features} Blue-tooth features\\
\label{Blue-tooth features}
\begin{lstlisting}
http://www.bluetooth.com/Pages/Basics.aspx
\end{lstlisting}



\bibitem{XML} Sample XML file in the format specified in DTD\\
\label{XML}
\begin{lstlisting}
http://forums.cs.adelaide.edu.au/file.php?file=%2F523%2Fsep2013-map.xml
\end{lstlisting}




\end{thebibliography}
\newpage



% section 2 Overall Description

\section{Overall Description}
\subsection{Product Perspective}
This software product consists of major two components.\\
\\
The first one is to run on the robot, a pre-existing LEGO Mindstorms NXT, for detecting and marking dangerous areas in the city.
 The second is the one that resides in the host system, which is capable of controlling the activities of the robot and displaying the mapping information.\\
\\
Continuous communication is required between the robot and host system and is achieved by means of Bluetooth device.\\
\subsection{Product Features}
The product consists of two components: the robot side and host side.
The main features of the robot side includes:

\begin{itemize}


\item The robot should be able to explore a city area and identify dangerous areas autonomously. After the robot arrives at the initial starting position of the city area, it would start to explore and map the site without intervention from the operator. 
\item The operator is allowed to commence manual control when necessary.
\item The robot should be able to mark the road closures, in the form of:\\
\\
A visual map constructed on paper/card;
\\
A visual representation in the map area of the GUI;
\\
textual representation XML file, which abide by the format specified by the DTD (i.e., Document Type Definition) file.
\item The robot should be able to stop autonomously for the sake of safety in the case of low battery level or losing communication with the host side.
\end{itemize}
The main features of the host side includes:
\begin{itemize}
\item The GUI (i.e., Graphical User Interface) on the computer allows the operator to manually control and monitor the robot's movement, which is achieved by a set of buttons and a map panel, respectively.
\item The operator is able to establish a communication with the robot by means of the GUI and Bluetooth device.
\item The operator is able to stop the movement of the robot using the emergency stop button in the GUI whenever an emergency occurs.
\end{itemize}



\subsection{User Classes and Characteristics}
Since programming techniques and technical know-how are not required, people from different fields are able to use this product after some basic training.
\\
In general, users can be classified into the following three groups:
\begin{itemize}
\item Trainer and Trainee
\\
For educational purposes, trainees can be motivated to create their own products for similar or distinguished critical tasks, after this product is demonstrated by the trainers.
\item Operator of Exploration Projects
\\
Operators of exploration projects can use this product to explore and map the area where it might be dangerous for human to be exposed, which would be more efficient and accurate than manually mapping by humans.
\end{itemize}


\subsection{Operating Environment}
The application will be installed on a pre-built Lego Mindstorms NXT robot, which has 256 Kbytes Flash and 64 Kbytes Ram on the leJOS 0.9.1 pre-flashed firmware. Also, the application shall be compatible with most operating systems (i.e MacOSX, Windows, Linux)



\subsection{Design and Implementation Constraints}
The design and implementation constraints originate from the pre-built robot hardware and client requirements. These constraints are listed as follows:%put in list formatting
\begin{itemize}
\item Hardware constraint. The Lego Robot has only 256 Kbytes to store the instructions, which is a restriction to development of the artificial intelligence of the robot.
\item Programming language constraint. The system must be implemented in Java lejos firmware.
\item The mapping data must be saved in the XML file abiding by the format defined in DTD.
\item The tool 'make' must be used to build any version of the software.
\item The software shall be tested in JUnit.
\item The project shall be documented using latex documentation language.
\end{itemize}

\subsection{User Documentation}

\begin{itemize}
\item API Documentation. The source code of the software will be documented in Javadoc to hyperlink related documents together, which would generate an API (i.e., Application Programming Interface) documentation in HTML format. This API Documentation would facilitate the developers and maintainers to maintain the source code.
\item User Manual. The user of the software will be provided with a User Manual, which contains the guide to the use of GUI and control of the robot.
\end{itemize}



\subsection{Assumptions and Dependencies}
\begin{itemize}
\item The communication between the robot and host computer is achieved by Bluetooth device, which has a maximum range of approximately 15 meters. We assume that the robot and the host computer would always be within this range during the mapping process.
\item We assume that the operator is able to monitor the movement of the robot clearly while it is mapping.
\end{itemize}






% section 3 User Requirements

\newpage



 \section{User Requirements}
 \subsection{Robot requirements} %either this or the latter is to be removed
\subsubsection{R01: Movement of the robot }
\begin{description}
\item[Description: ] \hfill \\ The robot shall be able to move forward or backward and have the ability to rotate left or right, on a plain surface.
The movement of the robot should be conducted in the following two conditions:

\begin{itemize}
\item The robot is moving under the operator’s control.
\item The robot is moving autonomously by means of intelligence developed for it within the boundaries of the city.
\end{itemize}

\item[Rationale: ] \hfill \\The robot is used to explore the city area, which requires the robot the ability to go forward or backword, rotate left or right.
\item[Acceptance Criteria: ] Run the program, connect the PC to the robot and press each movement direction button.
\item[Source: ] Project Description 2 Some Requirments.
\item[Priority: ] High
\end{description}


\subsubsection{R02: Obtaining current location of the robot }
\begin{description}
\item[Description: ] \hfill \\The robot shall be able to calculate the current location in the map, known as city area. This location information would be sent back to the host continuously in order to obtain the path this is transversed by the robot.

\item[Rationale: ] \hfill \\Both the robot and the operator need to ensure that the robot is in the range of city area. Therefore, information of the current location would help to avoid crossing the boundaries. In addition, the robot is required to mark a road closure or avoid obstacle on both the map in the form of paper, visual representation on the GUI and XML file, which needs the information of current location.
\item[Acceptance Criteria: ] Connect PC to the robot, load XML map file and move the robot to see robot changes its location accurately.
\item[Source: ] Project Description 2.1 The Map.
\item[Priority: ] High
\end{description}


\subsubsection{R03: Identifying Road Closures and Obstacles}
\begin{description}
\item[Description: ] \hfill \\ The robot shall be able to identify the road closures and obstacles with accordance to the data provided in the pre-defined or half-defined XML file, and mark them on the map, represented as paper.
\item[Rationale: ] \hfill \\The robot shall avoid to enter the dangerous areas in order to ensure its safety, which is of top priority.
\item[Acceptance Criteria: ] Connect PC to robot, run the robot on manual or automatic mode to expore and mark road closures by lowering down its marker to mark the closure and avoid obstacles.
\item[Source: ] Project Description 2.2 Map Representation for Prototype 
\item[Priority: ] High
\end{description}

\subsubsection{R04: Identifying boundaries of the map }
\begin{description}
\item[Description: ] \hfill \\ The robot shall be able to identify the boundaries of the city area and avoid wandering out of it, under both manual and automatic mode.
\item[Rationale: ] \hfill \\The robot is responsible for exploring the given city area. Crossing the boundaries could cause a potential hazard for the safety of the robot and make the mapping process less efficient.
\item[Acceptance Criteria: ] Run the robot to explore the map and check whether it stays withing the boundaries. The robot shall not cross outside boundaries.
\item[Source: ] Project Description  2.1 The Map.  
\item[Priority: ] High
\end{description}

\subsubsection{R05: Detecting obstacles }
\begin{description}
\item[Description: ] \hfill \\ Using ultrasonic sensor and bump sensor, the robot shall be able to detect and locate obstacles around it.
\item[Rationale: ] \hfill \\The city area to be explored may have many obstacles on the road. The robot equipped with ultrasonic sensor and bump sensor is able to detect these obstacles and have them marked on the map.
\item[Acceptance Criteria: ] Place an obstacle and run the robot into it to see the robot detecting the obstacle.
\item[Source: ] Project Description  2.1 The Map. 
\item[Priority: ] High
\end{description}

\subsubsection{R06: Detecting roads }
\begin{description}
\item[Description: ] \hfill \\ Within the city area, there are a number of roads, in which the robot is supposed to move. As roads are represented black lines, the robot can distinguish the roads by the changes of colors using light sensor.
\item[Rationale: ] \hfill \\In order to explore the city, the robot needs to transverse from point A to point B, which is achieved by moving along the road. Therefore, the robot should be capable of detecting roads using light sensor in order to accomplish its task.
\item[Acceptance Criteria: ] Place the robot on  road represented by a black line and run the robot. The robot shall not move out the road.
\item[Source: ] Client Meeting Minutes
\item[Priority: ] High
\end{description}

\subsubsection{R07: Detecting intersections }
\begin{description}
\item[Description: ] \hfill \\ Within the city area, there are a number of intersections which the robot needs to distinguish. These intersections are represented by black circles. The robot can distinguish the intersections by the changes of colors using light sensor.
\item[Rationale: ] \hfill \\In order to explore all paths in the city, the robot needs to know all intersections that it has traveled through. The robot can accomplish this task by identifying the black circle spots.
\item[Acceptance Criteria: ] run the robot on an intersection to see the robot choses a certain path
\item[Source: ] Client Meeting Minutes
\item[Priority: ] High
\end{description}

\subsubsection{R08: Marking road closures}
\begin{description}
\item[Description: ] \hfill \\ Road closures shall be marked under the following two circumstances:

\begin{itemize}
\item Road closures shall be displayed on the map if they are already pre-defined in the XML file.
\item The robot shall mark the road closure whenever it detects one.
\end{itemize}

When the robot marks the road closure, it needs to mark in the form of:

\begin{itemize}
\item A visual map constructed on paper/card;
\item A visual representation in the map area of the GUI
\item textual representation XML file, which abide by the format specified by the DTD (i.e., Document Type Definition) file.
\end{itemize}

\item[Rationale: ] \hfill \\The city struck by natural disaster leaves several area unsafe. Marking road closures could help to prevent people from moving into these areas.
\item[Acceptance Criteria: ] Run the robot into a pre-defined road closure for the robot to mark road closure on the map
\item[Source: ] Project Description  2.1 The Map. 
\item[Priority: ] High
\end{description}


\subsubsection{R09: Accepting commands }
\begin{description}
\item[Description: ] \hfill \\ The robot shall be able to perform commands from the operator at all times. 

\begin{itemize}
\item Under manual control mode, the robot should be able to accept and perform the command from the operator, such as forward, backward and rotate.
\item Under automatic control mode, the operator normally do not intervene in the process of mapping. However, the operator can interrupt the automatic mode by giving manual commands to the robot when necessary.
\end{itemize}

\item[Rationale: ] \hfill \\The communication between the host and the robot is represented as giving and accepting commands.
\item[Acceptance Criteria: ] Press on any buttons in the GUI. The robot shall respond to the pressing by performing the right action.
\item[Source: ] Project Description 2.3 Operation 
\item[Priority: ] High
\end{description}



\subsubsection{R10: Automated mapping }
\begin{description}
\item[Description: ] \hfill \\After manually directed to the initial starting position, the robot should start mapping automatically. The process of mapping will be terminated or interrupted under the following circumstances, respectively:
\begin{itemize}
\item The mapping is completed
\item Unexpected event occurs, such as loss of communication with the host or wandering out of the city boundaries
\end{itemize}
\item[Rationale: ] \hfill \\ Automated mapping could help to reduce time, cost and effort required to accomplish the task than manual control mode.
\item[Acceptance Criteria: ] Run the robot on the automatic mode. The robot shall start exploring independentlly.
\item[Source: ] Project Description  2.1 The Map. 
\item[Priority: ] High
\end{description}






\subsection{Mapping requirements} 
 \label{sec:Map}
\subsubsection{M01: Displaying transversed path }
\begin{description}
\item[Description: ] \hfill \\During the process of mapping, the transversed path of the robot shall be displayed on the GUI of the host side.

\item[Rationale: ] \hfill \\The transversed path displayed on the GUI could facilitate the operator to trace the history movement record of the robot.
\item[Acceptance Criteria: ] Run the robot on the map to follow certain paths. The transversed path of the robot shall be displayed on the GUI of the host side
\item[Source: ] Project Description  2.1 The Map. 
\item[Priority: ] High
\end{description}

\subsubsection{M02: Map Saving }
\begin{description}
\item[Description: ] \hfill \\During the process of mapping, the obtained mapping information, such as road closures and obstacles, as well as the transversed path information would be sent to the host side and stored in the form of XML file abiding by the format specified in the DTD.
\item[Rationale: ] \hfill \\The result of mapping should be stored in a document format such that the map could be reused.
\item[Acceptance Criteria: ] Chose the save option from the file options to save a currenntly explored map.
\item[Source: ] Project Description  2.1 The Map. 
\item[Priority: ] Medium
\end{description}


\subsubsection{M03: Map Loading }
\begin{description}
\item[Description: ] \hfill \\The host system should be able to open a saved map in the form of XML file in the format defined in the DTD. The opened map would be displayed on the GUI of the host side.
\item[Rationale: ] \hfill \\The host system should be able to reuse the map that is previously saved. In addition, the map can be updated during the mapping process, if the map was partially completed before.
\item[Acceptance Criteria: ] Chose the load option from the file options to load XML file map from the PC
\item[Source: ] Project Description  2.1 The Map. 
\item[Priority: ] Medium
\end{description}


\subsubsection{M04: Marked road closures and obstacles in XML }
\begin{description}
\item[Description: ] \hfill \\Information in relation to road closures and obstacles would be recorded in the form of textual representation XML file, which abides by the format specified by the DTD.
\item[Rationale: ] \hfill \\Road closures and obstacles are the main objects the robot is required to detect if they are not already defined in XML file. Therefore, these information should be stored in an appropriate document.
\item[Acceptance Criteria: ] Chose the save option from the file options to save a currenntly explored map. The map shall be saved in XML file format in the PC
\item[Source: ] Project Description  2.1 The Map. 
\item[Priority: ] High
\end{description}



\subsection{Host requirements}
\subsubsection{H01: GUI }
\begin{description}
\item[Description: ] \hfill \\ The program shall provide a graphical user interface which allows the operator to control and monitor the movement of the robot. The GUI should contain a number of buttons to perform the communication with the robot, such as connect to robot, movement directions control, mark road closure, stop. In addition, the GUI also contains a map panel that would display the mapping process under both the manual control mode and automatic mode. Finally, the GUI provides a list menu that allows the operator to save and load map in the form of XML file.
\item[Rationale: ] \hfill \\GUI is an important interface for communication between the operator and the robot. In addition, the map displayed on the GUI could also help the operator to monitor the movement of the robot.
\item[Acceptance Criteria: ] Run the program for the user interface to be generated with all buttons needed
\item[Source: ] Client Meeting Minutes
\item[Priority: ] High
\end{description}



\subsubsection{H02: Manual Control }
\begin{description}
\item[Description: ] \hfill \\ The operator shall be able to control the movement of the robot under the manual control mode by means of GUI, which contains a number of buttons that could control the robot movement and other features, such as mark road closure, return to base, and stop.
\item[Rationale: ] \hfill 
\begin{itemize}
\item Under manual control mode, remote control allows the operator to control the movement of the robot in order to accomplish some certain task, such as arriving at the initial start position. 

\end{itemize}
\item[Acceptance Criteria: ] Run the robot and press any manual control button. The robot shall respond to the pressing by performing the right action.
\item[Source: ] Project Description 2.3 Operation 
\item[Priority: ] High
\end{description}


\subsubsection{H03: Displaying the robot current position on GUI }
\begin{description}
\item[Description: ] \hfill \\The GUI on the host side would provide a map panel that displays a visual representation of the map with the robot current position shown on it.
\item[Rationale: ] \hfill \\ The operator should monitor the robot position at all times in case that the robot crosses the boundaries or enters a dangerous area.
\item[Acceptance Criteria: ] Run the robot on the physical map with XML map loaded on GUI. The robot current position shall be displayed accurately.
\item[Source: ] Client Meeting Minutes
\item[Priority: ] High
\end{description}

\subsubsection{H04: Emergency Stop }
\begin{description}
\item[Description: ] \hfill \\Under automatic mode, the operator can stop the process of mapping when observing an emergency.
\item[Rationale: ] \hfill \\ When unexpected event occurs, such as bumping in a obstacle, the operator can choose to stop the process of mapping in order to ensure the safety of the robot.
\item[Acceptance Criteria: ] Run the robot and press the button "STOP". The robot shall respond to the pressing by stoping any action.
\item[Source: ] Project Description 2.4 Safety
\item[Priority: ] High
\end{description}

\subsubsection{H05: Bluetooth Communication }
\begin{description}
\item[Description: ] \hfill \\The host system should be programmed in such a way that the Bluetooth communication is enabled.
\item[Rationale: ] \hfill \\ The communication between the robot and the host is achieved by Bluetooth device. Therefore, it is necessary to implement Bluetooth communication in order to establish communication between them.
\item[Acceptance Criteria: ] Turn on the robot and press connect button in the GUI. Wait for few seconds for the connection to be established
\item[Source: ] Client Meeting Minutes
\item[Priority: ] High
\end{description}

\subsubsection{H06: Alert Message }
\begin{description}
\item[Description: ] \hfill \\ Once the robot is enforced to stop due to the event of connection interruption or low battery power, one alert message is going to be sent to the operator telling the inferior circumstances of the robot. On the host side, the message will be shown on the GUI so the operator can reflect accordingly.

\item[Rationale: ] \hfill \\ The robot should have a safe way to deal with the occurrence of inferior situations such as low battery power and connection interruption. After the emergency stop to make the robot remain safe, the operator should have a way to know the information so actions can be taken to make the robot work again.
\item[Acceptance Criteria: ] Plug out the bluetooth device for alert message to be displayed in GUI.
\item[Source: ] Client Meeting Minutes
\item[Priority: ] Medium
\end{description}


\newpage




% section 4 System Features




\section{System Features}
\subsection{Manual Control}
\subsubsection{Description}
\begin{description}
\item[Priority: ] High
\end{description}
The system allows the operator to manually control the movement of the robot to perform the basic movements, such as moving forward or backward, rotating left or right, stop, etc. The manual control can be conducted under manual control mode, which are interchangeable by means of the buttons on the GUI.

\subsubsection{Stimulus/Response Sequences}
After the Bluetooth connection between the robot and the host side is established, the operator can control the movement of the robot by means of the buttons on the GUI. During the process of automatic mapping, the automated behaviour of the robot can be halted by any operation of the operator.

\subsubsection{Requirements}
The details of Functional requirements of this feature are described in Section 3.3 as in number H02.


\subsection{Automatic Mapping}
\subsubsection{Description}
\begin{description}
\item[Priority: ] High
\end{description}
Once the automatic mode is selected, the robot is able to perform the process of mapping automatically without the intervention of the operator.

\subsubsection{Stimulus/Response Sequences}
To perform automatic mapping, the operator should press the corresponding button to activate the robot’s automatic mode. Then the robot would commence the mapping procedures with accordance to algorithms developed for it. During the process of mapping, the robot would mark the obstacles and road closures in the means specified in 3.1 as in number R08. After the mapping process is completed, the operator is able to save the result of mapping in the means specified in 3.2 as in number M02.

\subsubsection{Requirements}
The details of Functional requirements of this feature are described in Section 3.1 as in number R10.




\newpage

% section 5 External Interface Requirements



\section{External Interface Requirements}
\subsection{User Interfaces}
A user interface is provided for the operator to manually control and monitor the movement of the robot. The user interface should consist of four parts: command buttons, robot information area, map area and list menu. The details of these parts are listed as follows:
\begin{itemize}
\item Command buttons. A number of buttons that allows the operator to control the movement of the robot. The command buttons include: forward, backward, left, right, stop, connection (i.e., to establish the connection with the robot), automatic mapping. The robot would accept and perform the corresponding commands when the operator presses these buttons.
\item Robot information area. This area contains the information in relation to the current battery level of the robot, the distance between the robot and objects in the front of it, and connection status of the robot.
\item Map area. The current map would be displayed in this area along with the path that the robot transversed. In addition, the current location of the robot would also be shown within the map.
\item List menu. List menu contains the functions that allow the operator to save and reload the map in the form of XML file in the format specified by the DTD.
\end{itemize}
\subsection{Hardware Interfaces}
As specified in Section 2.4, a pre-built Lego Mindstorms NXT robot would be provided to serve as the environment for running the software application. The robot has 256 Kbytes Flash and 64 Kbytes Ram on the leJOS 0.9.1 pre-flashed firmware, which also contains a Bluetooth module to allow the communication between the robot and the host.
\subsection{Software Interfaces}
The software application consists of two systems: robot control system and host system. The robot control system will be running on Lego Mindstorms NXT robot with the firmware of version leJOS 0.9.1, while the host system would require Java 6 or higher version platform.
\subsection{Communications interfaces}
The communication between the robot and the host side is achieved by means of the Bluetooth device without any encryption.

\newpage


% section 6 Use Cases


\section{Use Cases}
\subsection{UC001: Robot Movement Control}

\begin{description}
\item[Precondition: ] The robot must be turned on, connected to the PC and placed on the map.
\item[Basic Flow: ] \hfill
\begin{enumerate}
\item The robot is placed on the starting point of the map and is ready to recieve commands from the operator.
\item The operator presses on moving forwards button for the robot to move forward.
\item The robot moves forwards.
\item The operator releases the move forwards button.
\item The robot stops moving forwards.
\item The operator presses the rotating right button.
\item The robot rotates 90 degrees clockwise and stops.
\item The operator presses the rotating left button.
\item The robot rotates back 90 degrees anticlockwise and stops.
\item The operator presses on moving backwards button for the robot to move backwards.
\item The robot moves backwards.
\item The operator releases the move backwards button.
\item The robot stops moving backwards.
\end{enumerate}
\end{description}
 
\subsection{UC002: Map Loading and Saving}

\begin{description}
\item[Precondition: ] The robot must be turned on, connected to the PC and placed on the map.
\item[Basic Flow: ] \hfill
\begin{enumerate}
\item The operator choses the "Load" option from file menu options.
\item The operator browse for XML format map in PC.
\item The operator choses the XML file.
\item The operator open the XML file.
\item Map is loaded in the graphical user interface.
\item The robot is placed on the starting point of the map and is ready to recieve commands from the operator.
\item The operator presses on "Start Auto Mapping" button for the robot to start exploring.
\item The robot starts moving and exploring the map.
\item The robot marks road closures and obstacles.
\item The robot finishes exploration of the map and stops.
\item The operator choses the "Save" option from file menu options.
\item The operator locate where the map shall be saved  in PC.
\item The operator saves the XML file map.
\end{enumerate}
\end{description}

% section 7 Other Non-functional Requirements



\section{Other Non-functional Requirements}

%  Performance Requirements
\subsection{Performance Requirements}
\subsubsection{P01: Real Time Mapping}

\begin{description}
\item[Description: ] \hfill \\The system is required to construct the map in real-time during the process of mapping. The maximum delay can be accepted will be less than 0.5 seconds, as it is a safety-critical project.
\item[Rationale: ] \hfill \\With real time mapping, the operator can immediately observe any changes on the map in order to ensure the safety of the robot and completeness of mapping.
\item[Acceptance Criteria: ] Load XML map and run the robot. Mapping shall be parallel with the robot movement
\item[Source: ] 1\_client\_meeting\_minutes\_12-08-13
\item[Priority: ] Medium
\end{description}


%  Safety Requirements
\subsection{Safety Requirements}

\subsubsection{SA01: Moving Speed}
\begin{description}
\item[Description: ] \hfill \\The default speed of the robot shall be 5cm/s that is to ensure speed is safe for the robot
\item[Rationale: ] \hfill \\The robot shall not move fast, otherwise the robot could be damaged due to too much force by bumping in any obstacles. The speed option shall be between 0cm/s and 15cm/s.
\item[Acceptance Criteria: ] Run the robot and change the speed from the speed slider in GUI. The robot shall increase or decrease its speed accordinglly.
\item[Source: ] 1\_client\_meeting\_minutes\_12-08-13
\item[Priority: ] High
\end{description}

\subsubsection{SA02: Autonomously halt}
\begin{description}
\item[Description: ] \hfill \\The robot should be able to halt autonomously under the following two circumstances: \hfill 
\begin{itemize}
\item Communication interruption
\item Low battery power
\end{itemize}
After the autonomous halt, the robot should be able to send alert to the operator so the operator will notice the interrupted circumstance.
\item[Rationale: ] \hfill \\The city area to be explored could be very dangerous. When the above events occur, the best solution would have the robot autonomously halt.
\item[Acceptance Criteria: ] Plug out the bluetooth device for alert message to be displayed in GUI. Run the robot with a low battery level.
\item[Source: ] 1\_client\_meeting\_minutes\_12-08-13
\item[Priority: ] High
\end{description}


%  Security Requirements
\subsection{Security Requirements}
\subsubsection{SE01: Bluetooth Communication Security}
\begin{description}
\item[Description: ] \hfill \\ The connection between the robot and the host should be established after the operator inputs a 4 digits pin number.
\item[Rationale: ] \hfill \\ The LEGO Mindstorms NXT robot provides a Bluetooth pairing with 4 digits pin to avoid any unintentional connection.
\item[Acceptance Criteria: ] Before establishing a connection the robot shall ask the operator to enter 4 digits pin number for the connection to be established.
\item[Source: ]  
\item[Priority: ] Low
\end{description}


%  Software Quality Attributes
\subsection{Software Quality Attributes}

\subsubsection{SO01: Maintainability}
\begin{description}
\item[Description: ] \hfill \\ After being delivered, the product shall be easy to use and maintain for users.
\item[Rationale: ] \hfill \\ The user of the product may have new requirements that would need some modification on the system. Therefore, using the coding convention and appropriate documentation, the system would be easy to maintain.
\item[Acceptance Criteria: ] Ensure all deliverables are clearly explained and well understood 
\item[Source: ] 1\_client\_meeting\_minutes\_12-08-13
\item[Priority: ] Medium
\end{description}


\newpage



% Section 7 Other Requirements

\section{Other Requirements}
\subsection{Quality Requirements}
\subsubsection{O01: Delivery Requirement}
\begin{description}
\item[Description: ] \hfill \\ The allocated tasks should be finished and submitted on time with good quality.
\item[Rationale: ] \hfill \\ Each member in this group are endeavoring to deliver the best product for the client.
\item[Acceptance Criteria: ] NIL
\item[Source: ] Group Meeting Minutes 
\item[Priority: ] High
\end{description}

\newpage

%  appendix
\appendix
\section{Glossary}
\begin{description}
\item[API] Application Programming Interface
\item[DTD] Document Type Definition
\item[GUI] Graphical User Interface
\item[PIN] Personal Identification Number
\item[SDD] Software Design Document
\item[SEP] Software Engineering and Project
\item[SPMP] Software Project Management Plan
\item[SRS] Software Requirements Specification
\item[XML] Extensible Markup Language
\end{description}

\end{document}